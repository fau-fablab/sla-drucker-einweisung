% !TeX spellcheck = de_DE_frami
%%%%%%%%%%%%%%%%%%%%%%%%%%%%%%%%%%%%%%%%%%%%%%%%
% COPYRIGHT: (C) 2012-2015 FAU FabLab and others
% Bearbeitungen ab 2015-02-20 fallen unter CC-BY-SA 3.0
% Sobald alle Mitautoren zugestimmt haben, steht die komplette Datei unter CC-BY-SA 3.0. Bis dahin ist der Lizenzstatus aller alten Bestandteile ungeklärt.
%%%%%%%%%%%%%%%%%%%%%%%%%%%%%%%%%%%%%%%%%%%%%%%%


\newcommand{\basedir}{fablab-document}
\documentclass{\basedir/fablab-document}
\usepackage{amssymb} % Symbole für Knöpfe
\usepackage{subfigure,caption}
\usepackage{eurosym}
\usepackage{textcomp} % \textcelsius
\usepackage{tabularx} % Tabellen mit bestimmtem Breitenverhältnis der Spalten
\usepackage{wrapfig} % Textumlauf um Bilder
\usepackage{float} % Ermöglicht H als Platzierungsoption
\usepackage{array} 
\usepackage[hyphens]{url}

\renewcommand{\texteuro}{\euro}
\newcommand{\fachbegriff}[1]{(\textit{#1})}
\newcommand{\ts}[1]{\textsuperscript{#1}}
\newcommand{\ra}{$\Rightarrow$}

% \linespread{1.2}
% \fancyhead{}
\date{\today}
\author{kontakt@fablab.fau.de}
\title{Einweisung SLA 3D-Drucker}

\begin{document}
	
	\maketitle
	\begin{center}
		Für den 3D-Drucker \textbf{FormLabs Form2}
	\end{center}
	
	\textbf{Nur eingewiesene Benutzer dürfen den Drucker selbstständig benutzen}, um teure Beschädigungen zu vermeiden. Wenn du noch nicht eingewiesen bist, frage einen Betreuer. Er erklärt dir die Bedienung und lässt dich unter Aufsicht dein gewünschtes Teil ausdrucken. Wenn du alles verstanden hast, darfst auch du dann die Einweisung unterschreiben und den Drucker in Zukunft selbstständig verwenden.
	
	\section{Regeln und Hinweise}
	% Hier sollte nur das wichtigste gegen "Kaputtmachen" des Druckers stehen.
	Für die Benutzung ist es wichtig, dass du folgende Hinweise beachtest:
	
	\begin{itemize}
		\item Die Materialien sind gesundheitsschädlich -- \textbf{Schutzbrille und Nitrilhandschuhe tragen!}
		\item Obwohl das 3D-Drucken unter den Begriff ``Rapid Prototyping'' fällt, kann ein Druckteil je nach Größe und
		Präzision gut mehrere Stunden bis Tage dauern. Betreuer und die Software helfen dir, die Dauer abzuschätzen.
		\item Wenn du nicht bis zum Ende deines Drucks da sein kannst, frage vorher einen Betreuer und hinterlasse einen Zettel mit deinem Namen und deinen Kontaktdaten.
		\item Anleitung ist exakt zu beachten. Wenn du nicht weiter weißt oder dir unsicher bist, frag einen Betreuer.
		\item Drucker nach Beendigung des Druckes in Ruhezustand versetzen.
		\item Materialwechsel, Pflege und Wartung darf \textbf{nur von einem Betreuer} durchgeführt werden.
		\item \textbf{Bezahlen bitte nicht vergessen!}
	\end{itemize}
%\newpage
\subsection{Gefahren für Mensch und Maschine}
\begin{table}[h]
	\centering
	\begin{tabular}{ccc}
		
		\includegraphics[width=2cm]{bilder/GHSa.png}  &
		\includegraphics[width=2cm]{bilder/GHShs.jpg}  & \includegraphics[width=2cm]{bilder/GHSf.png}
	\end{tabular}
\end{table}

\begin{itemize}
	\item \textbf{Achtung:} Sowohl das verwendete Kunstharz als auch Isopropylalkohol sind gesundheitsschädlich. Bitte Haut- und Augenkontakt, Verschlucken und einatmen der Dämpfe vermeiden.
	\item \textbf{Achtung:} Sowohl beim Umgang mit dem Kunstharz also auch beim Umgang mit Isopropylalkohol für die Nachbehandlung des Druckteils sind Nitrilhandschuhe zu tragen. 
	\item \textbf{Achtung:} Isopropylalkohol ist leicht entflammbar. Rauchen und Umgang mit offenen Flammen und Zündquellen sind in der Nähe verboten. 
	\item \textbf{Achtung:} Der Harztankträger ist das Heizelement des Harzes und kann während des Druckens heiß werden.
	\item \textbf{Achtung:} Harztank nicht mit Deckel im Drucker lagern, bei Stromanschluss führt der Drucker einen Selbsttest durch und bewegt den Wischer.
	\item Ausdrucke sind \textbf{nicht} lebensmittelecht oder biokompatibel, weil sie  auch nach Nachbehandlung noch Spuren des Rohmaterials enthalten können. Trinkbecher oder Armbanduhren sind also keine gute Idee.
\end{itemize}

%%%%%%%%%%%

\subsection{Persönliche Schutzausrüstung}


\begin{table}[h]
	\centering
	\begin{tabular}{cc}
		
	 \includegraphics[width=2cm]{bilder/gaugenschutz.png}  & \includegraphics[width=2cm]{bilder/ghandschuh.png} \\
	\end{tabular}
\end{table}

Bei allen Arbeiten am Drucker und bei der Nachbearbeitung der Teile sind zu tragen:
\begin{itemize}
	\item \textbf{Schutzbrille} 
	\item \textbf{Nitril-Einweghandschuhe}
\end{itemize}
Erst nach Waschen und Nachbelichten dürfen die Teile auch ohne Handschuhe angefasst werden. Vorher können sich noch Reste des Rohmaterials an der Oberfläche befinden.
	\newpage

% % % % % % %
	
	\renewcommand{\contentsname}{Inhaltsverzeichnis / Arbeitsablauf}
	\setcounter{tocdepth}{2}
	\tableofcontents
	\newpage
	
	% % % % % % % % % % % % % % % %
	\section{3D-Modell erstellen}
	
	\subsection{Dateiformat}
	
	Im STL-Dateiformat, Einheit: Millimeter. Alle gängigen 3D-Programme haben einen STL-Export.
	
	\begin{itemize}
		\item auf \href{https://thingiverse.com}{Thingiverse.com} gibt es viele vorgefertigte Modelle, als
		Grundlage oder gleich zum fertig ausdrucken.
		\item oder erstelle ein Modell mit einem Programm deiner Wahl
		\begin{table}[H]
			\centering
			\begin{tabularx}{\textwidth}{|l|X|}
				\hline \textbf{Name} & \textbf{Beschreibung} \\
				\hline \multicolumn{2}{|c|}{\textit{kostenlose Software}}  \\
				\hline Blender & relativ komplex aber auch für Freiformflächen geeignet  \\
				\hline OpenSCAD & Skriptsprache für Konstruktion aus geometrischen Grundkörpern \\
				\hline DesignSpark Mechanical & Angelehnt an professionelle CAD-Software, aber relativ einfach zu bedienen  \\
				\hline TinkerCAD & sehr einfach, für Kinder gut geeignet  \\
				\hline Google SketchUp & wenig Einarbeitung, geringer Funktionsumfang, für einfache Teile \\
				\hline & \\
				\hline \multicolumn{2}{|c|}{\textit{kostenpflichtige Software (proprietär)}}  \\
				\hline PTC Creo, Solid Edge, Siemens NX & kostenlos beim RRZE für Studenten, professionelle Software \\
				\hline Autocad Inventor & kostenlos bei Autodesk für Studenten ebenfalls für professionelle Anwendungen \\
				\hline
			\end{tabularx}
		\end{table}
	\end{itemize}
	
	\subsection{Einschränkungen der Formen}
	\begin{itemize}
		\item \textbf{Bauraum}\\L x B x H: 145\,mm x 145\,mm x 175\,mm
		\item \textbf{Große Druckteile}\\Aufgrund der langen Druckdauer und bisher fehlender Erfahrung mit großen Drucken, empfehlen wir Objekte kleiner als 100x100x50\,$\mathrm{mm}^3$ (L x B x H).
		\item \textbf{Stützstrukturen}\\Damit Druckteile sinnvoll gedruckt werden können, müssen Stützstrukturen verwendet werden. Die Software erzeugt hierbei ein loses Geflecht an Verbindungen zu Überhängen und Brücken, die sich nach dem Ausdrucken mit einer Zange oder einem Skalpell entfernen lassen.
		\item \textbf{Ablösewanne}\\Durch die Verwendung von Stützstrukturen wird automatisch eine sogenannte Ablösewanne unter dem Objekt generiert. Diese erleichtert das Ablösen von der Buildplattform. Auf dem Rand der Ablösewanne wird zusätzlich der Dateiname der STL-Datei gedruckt.
	\end{itemize}

\subsection{Designrichtlinien für SLA Drucke}

Die folgenden Designrichtlinien helfen dir dein Druckteil für den $Form2$ zu optimieren und beschreiben welche Mindestwerte für den Formlabs-Drucker empfohlen werten. Obwohl das über- bzw. unterschreiten der Werte nicht verboten ist, empfiehlt es sich, dies nur zu tun wenn es ein konkreter Grund erzwingt, und auch dann nur mit Bedacht.  

\textbf{Hinweis:} Die folgenden Richtlinien gelten für das Drucken mit dem transparenten Kunstharz bei 100 $\mathrm{\mu}m$. Bei anderen Kunstharzen und anderen Schichtdicken gibt es möglicherweise Abweichungen.

	\begin{itemize}
	\item \textbf{Mindestdicke von gestützten Wänden:}  0,4 mm\\
	\\
	\textbf{Hinweis:} Gehe beim Nachbearbeiten von dünnen Wänden vorsichtig vor, da diese das Isopropanol absorbieren und anschwellen können. Dies kann zu Verformungen führen, deshalb das Druckteil nur kurz zum waschen eintauchen. \\
	\item \textbf{Mindestdicke nicht gestützter Wände:}  0,6 mm
	\item \textbf{Maximale Länge nicht gestützter Überhänge:}  1,0 mm
	\item \textbf{Mindestwinkel für nicht gestützte Überhänge:}  19$^\circ$ von der Ebene
	\item \textbf{Maximale horizontale Stützbrücke:}  21 mm
	\item \textbf{Mindestdurchmesser von vertikalen Drähten:}  1,5 mm \\
	\\
	\textbf{Hinweis:}Gehe beim Nachbearbeiten von vertikalen Drähten vorsichtig vor, da diese das Isopropanol absorbieren und anschwellen und geschwächt werden. Dies kann zu Verformungen oder dem Abbruch des Drahtes führen, deshalb das Modell nur kurz zum waschen eintauchen. \\
	\item \textbf{Mindestwert für geprägte Details:}  0,1 mm
	\item \textbf{Mindestwert für eingravierte Details:}  0,4 mm
	\item \textbf{Mindestabstand zwischen mehreren Druckteilen:}  1 mm
	\item \textbf{Mindestdurchmesser für Aussparungen:}  0,5 mm
	\item \textbf{Hohlkörper} \\
	Abflussmöglichkeiten für das verbleibende Kunstharz vorsehen, da dieses im Inneren verbleibt. Aufgrund der Viskosität des Harzes sollten diese Öffnungen einen Durchmesser von mindestens 3,5\,mm aufweisen.
	
\end{itemize}


	\newpage
	
%%%%%%%%%%%%


\section{3D-Modell mit PreForm umwandeln und ausdrucken}
 
 \subsection{Preform Vorbereitung}
 
	\begin{itemize}
		\item Vor dem Druckt ist es notwendig,n die STL-Datei mit einem eindeutigen Namen zu versehen, zum Beispiel "Max\_Mustermann\_E-Mail\_Dateiname"\ .
		\item Programm "PreForm" öffnen und Drucker "PrettyTiger"\, auswählen und mit "Auswählen" bestätigen.
		\item Mit Klick auf den Dialog "Datei"\  und anschließend mit "Öffnen"\ STL-Datei öffnen.
		\item Bei Bedarf das Druckteil mit den Schaltflächen skalieren, drehen und verschieben.
		\item Notwendige Stützstrukturen automatisch erstellen lassen, bei Bedarf händisch anpassen.
	\end{itemize}

\subsection{Wahl der Schichtdicke}

\begin{table} [H]
	\centering
	\begin{tabular}{|c|c|c|c|}\hline
	  & 100\,$\mu m$ & 50\,$\mu m$ & 25\,$\mu m $\\ \hline
	Transparent & \checkmark &\checkmark &\checkmark \\ \hline
	Flexibel & \checkmark &\checkmark & X\\ \hline
	\end{tabular}
\end{table}

Die Wahl der Schichtdicke ist entscheidend für die Dauer der Druckzeit des Druckteils. Im Gegensatz zu FDM-3D-Druckern (z.B. Ultimaker) ist die Druckqualität kaum von der Schichtdicke abhängig (außer bei sehr kleinen Objekten). Unsere Empfehlung ist daher die 100\,$\mu m$ Schichtdicke zu verwenden. \\
Im Vergleich zu 100\,$\mu m$ benötigt ein Druckteil in der Auflösung von 50\,$\mu m$ mindestens doppelt so lange, und ein Druck in der Schichtdicke  25\,$\mu m$ (falls verfügbar) dauert mindestens vier mal so lange.
		 	
\subsection{Drucken}
 \begin{itemize}
 		
		\item Zum Drucken des Objektes die orange Schaltfläche "Druckeinrichtung"\ öffnen.\\
		\item \textbf{Hinweis:} Notiere dir bitte die nun angezeigte kalkulierte Menge an Harz, die für dein Druckteil benötigt wird. Diese ist notwendig, um deinen Druck bezahlen zu können.
		\item In diesem Dialog ist vor dem Senden der Name des Druckteils zu deklarieren. Hier bitte mindesens Name und E-Mail-Adresse angeben, am besten auch noch die Telefonnummer.\\
		\item Nach erfolgreichem Senden befindet sich das Druckteil nun auf dem Drucker selbst. Hierzu einfach den Anweisungen auf dem Bildschirm des $Form2$ folgen bis der Druck gestartet ist.
		\item Wenn du nicht bis zum Ende deines Drucks da sein kannst, frage vorher einen Betreuer und hinterlasse einen Zettel mit deinem Namen und deinen Kontaktdaten.
		
\end{itemize}

	\pagebreak
	
	%%%%%%%%%%%%%%%%%%
	
\section{Nachbereitung}

Nach erfolgreichem Druck kann nun mit der Nachbearbeitung des Druckteils begonnen werden. 

\textbf{Hinweis:} Ab diesen Zeitpunkt muss die vorgeschriebene Schutzausrüstung (Schutzbrille, Handschuhe) getragen werden.

	\begin{enumerate}
	\item Die ganze Konstruktionsplattform lösen.
	\begin{itemize}
		\item Hebel an der Oberseite lösen und Konstruktionsplattform zu sich hin ziehen und herausnehmen.
		\item Orangene Haube des Druckers wieder verschließen, sonst kann es zu Verunreinigungen des Harzes im Tank kommen.
	\end{itemize}
	\item Konstruktionsplattform mit dem gedruckten Teil nach oben in die Haltevorrichtung einschieben.
	\item Entfernen des Druckteils mit dem Ablösewerkzeug.
	\begin{itemize}
		\item Mit dem Ablösewerkzeug unter die angewinkelte Kante der Basis und anschließend hebeln bis sich der Druck löst.
		\item \textbf{Hinweis:} Bitte dabei möglichst schonend mit der Kostruktionsplattform umgehen.
		\item Ist das Druckteil gelöst muss die Kostruktionsplattform wieder in den Drucker eingesetzt werden.
	\end{itemize}
	\item Das gelöste Druckteil im Bad 1 schwenken
	\begin{itemize}
		\item max. 30\,Sekunden
	\end{itemize}
	\item Das Druckteil im Bad 2 einweichen
	\begin{itemize}
		\item max. 10\,Minuten
		\item \textbf{Hinweis:} Bei filigranen Teilen diesen Schritt überspringen
	\end{itemize}
	\item Nachhärtung
		\begin{itemize}
		\item Das Druckteil wird durch den UV-Anteil im Umgebungslicht über die nächsten 24 Stunden komplett aushärten 
		\item Bei Bedarf kann man dieses Aushärten mit Hilfe von UV-Licht beschleunigen, hierbei reichen ca. 15\,Minuten im Belichter
		\item Erst danach ist die Oberfläche des Drucks als \enquote{ungefährlich} anzusehen.
	\end{itemize}
	\item Entfernen der Stützstrukturen
	
	\end{enumerate}
	
	%%%%%%%%%%%%%%%%%
	
	\section{Bezahlen und abschließen}
	
	Nach der Nachbearbeitung des Druckteils ist die Arbeitsfläche aufzuräumen und dem Nächsten den Platz frei zu machen, und, sofern noch nicht geschehen, das verbrauchte Harzvolumen am Drucker abzulesen und zu notieren. Falls keine weitere Person drucken will ist der $Form2$ in Ruhezustand zu versetzen. Dies geschieht durch Drücken und Halten der Drucktaste für 6-10 Sekunden, bis die Anzeige erlischt. \\ 
	\\
	Abgerechnet wird das Druckteil am Kassenterminal, hierzu ist das abgelesene Volumen Harz in $ml$ zu kennen, und am Kassenterminal auf ganze $ml$ aufgerundet zu bezahlen.
	\pagebreak
	
	% % % % % % % % % % % %

	\section{Pflege \& Wartung \textcolor{red}{Nur von Betreuer durchzuführen}}
	
	\subsection{Ausrichten des Druckers}
	Der \textit{Form2} muss, um fehlerfrei drucken zu können, nivelliert sein. Hierzu gibt es ein Eintrag im Einstellungsmenü des Druckers.\\
	Ist der Drucker nicht nivelliert, so ist dieser Menüeintrag auszuwählen und den Anweisungen auf dem Bildschirm zu folgen. Zum justieren wird das runde Nivellierwerkzeug benötigt. Dieses schiebt man unter die jeweilige Ecke und dreht es im Uhrzeigersinn um den Drucker tiefer, und gegen den Uhrzeigersinn um ihn höher zu stellen.\\
	Weitere Informationen dazu auf der Supportseite des Herstellers.	\\
	 \url{https://support.formlabs.com/hc/de/articles/115000013164-Den-Form-2-nivellierens}
	 
	 \subsection{Pflege der Harzkartusche}
	 Die Harzkartusche muss ungefähr alle zwei Wochen geschüttelt werden, damit das Kunstharz stets gut durchmischt ist. Dadurch erhöht sich die Druckqualität.
	
	\subsection{Harztank}
		Der Harztank sollte regelmäßig von gereinigt werden. Anleitung dazu auf der Supportseite des Herstellers\\
		\url{https://support.formlabs.com/hc/de/articles/115000016184-Pflege-des-Harztanks}
		
		\subsection{Glasfenster}
		Zur Reinigung des Glasfensters unter dem Harztanks bei Verschmutzung durch Schlieren oder Fingerabdrücken siehe:\\ \url{https://support.formlabs.com/hc/de/articles/115000016204-Reinigung-des-optischen-Glasfensters}
	
	\subsection{Firmwareupdates}
	
In regelmäßigen Abständen kommt es vor, dass die Firmware des Druckers zu updaten ist. Eine Anleitung findet sich hierbei auf der Website des Herstellers.\\	
		\url{https://support.formlabs.com/hc/de/articles/115000013690-Firmwareupdate-durchführen}
	
\end{document}
