% !TeX spellcheck = de_DE_frami
%%%%%%%%%%%%%%%%%%%%%%%%%%%%%%%%%%%%%%%%%%%%%%%%
% COPYRIGHT: (C) 2012-2015 FAU FabLab and others
% Bearbeitungen ab 2015-02-20 fallen unter CC-BY-SA 3.0
% Sobald alle Mitautoren zugestimmt haben, steht die komplette Datei unter CC-BY-SA 3.0. Bis dahin ist der Lizenzstatus aller alten Bestandteile ungeklärt.
%%%%%%%%%%%%%%%%%%%%%%%%%%%%%%%%%%%%%%%%%%%%%%%%


\newcommand{\basedir}{fablab-document}
\documentclass{\basedir/fablab-document}
\usepackage{amssymb} % Symbole für Knöpfe
\usepackage{subfigure,caption}
\usepackage{eurosym}
\usepackage{textcomp} % \textcelsius
\usepackage{tabularx} % Tabellen mit bestimmtem Breitenverhältnis der Spalten
\usepackage{wrapfig} % Textumlauf um Bilder
\usepackage{float} % Ermöglicht H als Platzierungsoption
\usepackage{array} 
\usepackage[hyphens]{url}

\renewcommand{\texteuro}{\euro}
\newcommand{\fachbegriff}[1]{(\textit{#1})}
\newcommand{\ts}[1]{\textsuperscript{#1}}
\newcommand{\ra}{$\Rightarrow$}

% \linespread{1.2}
% \fancyhead{}
\date{\today}
\author{kontakt@fablab.fau.de}
\title{Einweisung SLA 3D-Drucker}

\begin{document}
	
	\maketitle
	\begin{center}
		Für den 3D-Drucker \textbf{FormLabs Form2}
	\end{center}
	
	\textbf{Nur eingewiesene Benutzer dürfen den Drucker selbstständig benutzen}, um teure Beschädigungen zu vermeiden. Wenn du noch nicht eingewiesen bist, frage einen Betreuer. Er erklärt dir die Bedienung und lässt dich unter Aufsicht dein gewünschtes Teil ausdrucken. Wenn du alles verstanden hast, darfst auch du dann die Einweisung unterschreiben und den Drucker in Zukunft selbstständig verwenden.
	
	\section{Regeln und Hinweise}
	% Hier sollte nur das wichtigste gegen "Kaputtmachen" des Druckers stehen.
	Für die Benutzung ist es wichtig, dass du folgende Hinweise beachtest:
	
	\begin{itemize}
		\item Obwohl das 3D-Drucken unter den Begriff ``Rapid Prototyping'' fällt, kann ein Druck je nach Größe und
		Präzision gut mehrere Stunden bis Tage dauern. Betreuer als auch die druckerspezifische Software helfen dir, die Dauer abzuschätzen.
		\item Nicht unbeaufsichtigt drucken, immer wieder mal einen Blick darauf werfen, besonders am Anfang.\\
		Wenn du nicht bis zum Ende deines Drucks da sein kannst, frage vorher einen Betreuer und hinterlasse einen Zettel mit Name und Kontaktdaten.
		\item Anleitung exakt beachten. Wenn du nicht weiter weißt oder dir unsicher bist, frag einen Betreuer.
		\item Drucker nach Beendigung des Druckes in Ruhezustand versetzen.
		\item Materialwechsel, Pflege und Wartung darf nur von einem Betreuer durchgeführt werden.
		\item \textbf{Wiegen und Bezahlen nicht vergessen!}
	\end{itemize}

\section{Warnhinweise}
\begin{table}[h]
	\centering
	\begin{tabular}{ccc}
		
		\includegraphics[width=2cm]{bilder/GHSa.png}  &
		\includegraphics[width=2cm]{bilder/GHShs.jpg}  & \includegraphics[width=2cm]{bilder/GHSf.png} \\
	\end{tabular}
\end{table}

\begin{itemize}
	\item \textbf{Achtung:} Das verwendete Kunstharz ist schädlich. Bitte Haut- und Augenkontakt vermeiden \\
	\item \textbf{Achtung:} Beim Umgang mit Isopropylalkohol für die Nachbehandlung der 3D-Druckteile. Dieses ist schädlich für die Haut und leicht Entflammbar\\
	
	\item \textbf{Achtung:} Der Harztankträger ist das Heizelement des Harzes und kann während des Druckens sehr heiß werden.
\end{itemize}

\section{Persönliche Schutzausrüstung}
\begin{table}[h]
	\centering
	\begin{tabular}{cc}
		
	 \includegraphics[width=2cm]{bilder/gaugenschutz.png}  & \includegraphics[width=2cm]{bilder/ghandschuh.png} \\
	\end{tabular}
\end{table}

\begin{itemize}
	\item Schutzbrille, unerlässlich bei der Nachbearbeitung der 3D-Druckteile \\
	\item Schutzhandschuhe bei der Nachbearbeitung der 3D-Druckteile\\
\end{itemize}
	\newpage

% % % % % % %
	
	\renewcommand{\contentsname}{Inhaltsverzeichnis / Arbeitsablauf}
	\setcounter{tocdepth}{2}
	\tableofcontents
	\newpage
	
	% % % % % % % % % % % % % % % %
	\section{3D-Modell erstellen}
	\subsection{Dateiformat}
	
	Im STL-Dateiformat, Einheit: Millimeter. Alle gängigen 3D-Programme haben einen STL-Export.
	
	\begin{itemize}
		\item auf \href{https://thingiverse.com}{Thingiverse.com} gibt es viele vorgefertigte Modelle, als
		Grundlage oder gleich zum fertig ausdrucken.
		\item oder erstelle es mit einem Programm deiner Wahl
		\begin{table}[H]
			\centering
			\begin{tabularx}{\textwidth}{|l|X|}
				\hline \textbf{Name} & \textbf{Beschreibung} \\
				\hline \multicolumn{2}{|c|}{\textit{kostenlose Software}}  \\
				\hline Blender & relativ komplex aber auch für Freiformflächen geeignet  \\
				\hline OpenSCAD & Skriptsprache für Konstruktion aus geometrischen Grundkörpern \\
				\hline DesignSpark Mechanical & Angelehnt an professionelle CAD-Software, aber relativ einfach zu bedienen  \\
				\hline TinkerCAD & sehr einfach, für Kinder gut geeignet  \\
				\hline Google SketchUp & wenig Einarbeitung, geringer Funktionsumfang, für einfache Teile \\
				\hline & \\
				\hline \multicolumn{2}{|c|}{\textit{kostenpflichtige Software (proprietär)}}  \\
				\hline PTC Creo, Solid Edge, Siemens NX & kostenlos beim RRZE für Studenten, professionelle Software \\
				\hline Autocad Inventor & kostenlos bei Autodesk für Studenten ebenfalls für professionelle Anwendungen \\
				\hline
			\end{tabularx}
		\end{table}
	\end{itemize}
	
	\subsection{Einschränkungen der Formen}
	\begin{itemize}
		\item \textbf{Bauraum}\\LxBxH: 145\,mm x 145\,mm x 175\,mm
		\item \textbf{Große Teile}\\Aufgrund der langen Druckdauer und bisher fehlender Erfahrung mit großen Drucken, empfehlen wir Objekte kleiner als 100x100x50\,$\mathrm{mm}^3$ (LxBxH).
		\item \textbf{Stützstrukturen}\\Damit Teile sinnvoll gedruckt werden können, müssen Stützstrukturen verwendet werden. Die Software erzeugt hierbei ein loses Geflecht an Verbindungen zu Überhängen und Brücken, die sich nach dem Ausdrucken mit einer Zange oder einem Skalpell entfernen lassen.
		\item \textbf{Ablösewanne}\\Durch die Verwendung von Stützstrukturen wird automatisch eine sogenannte Ablösewanne unter dem Objekt generiert. Diese erleichtert das Ablösen von der Buildplattform. Auf dem Rand der Ablösewanne wird zusätzlich der Dateiname gedruckt.
	\end{itemize}

\subsection{Designrichtlinien für SLA Drucke}

Die folgenden Designrichtlinien helfen dir dein 3D Druck für den Form2 zu optimieren und welche Mindestvoraussetzungen für den Formlabs-Drucker gelten.

\textbf{Hinweis:} Die folgenden Richtlinien gelten für das Drucken mit dem transparenten Kunstharz bei 100 $\mathrm{\mu}m$. Bei anderen Kunstharzen und anderen Schichtdicken gibt es möglicherweise Abweichungen.

	\begin{itemize}
	\item \textbf{Mindestdicke von gestützten Wänden:}  Empfohlen: 0,4 mm\\
	\\
	\textbf{Hinweis:}Gehe beim Nachbearbeiten von dünnen Wänden vorsichtig vor, da diese das Isopropanol absorbieren und anschwellen können. Dies kann zu Verformungen führen, deshalb das Modell nur kurz zum waschen eintauchen. \\
	\item \textbf{Mindestdicke nicht gestützter Wände:}  Empfohlen: 0,6 mm
	\item \textbf{Maximale Länge nicht gestützter Überhänge:}  Empfohlen: 1,0 mm
	\item \textbf{Mindestwinkel für nicht gestützte Überhänge:}  Empfohlen: 19$^\circ$ von der Ebene
	\item \textbf{Maximale horizontale Stützbrücke:}  Empfohlen: 21 mm
	\item \textbf{Mindestdurchmesser von vertikalen Drähten:}  Empfohlen: 1,5 mm \\
	\\
	\textbf{Hinweis:}Gehe beim Nachbearbeiten von vertikalen Drähten vorsichtig vor, da diese das Isopropanol absorbieren und anschwellen und geschwächt werden. Dies kann zu Verformungen oder dem Abbruch des Drahtes führen, deshalb das Modell nur kurz zum waschen eintauchen. \\
	\item \textbf{Mindestwert für geprägte Details:}  Empfohlen: 0,1 mm
	\item \textbf{Mindestwert für eingravierte Details:}  Empfohlen: 0,4 mm
	\item \textbf{Mindestabstand zwischen Objekten:}  Empfohlen: 1 mm
	\item \textbf{Mindestdurchmesser für Aussparungen:}  Empfohlen: 0,5 mm
	\item \textbf{Hohlkörper} \\
	Abflussmöglichkeiten für das verbleibende Kunstharz vorsehen, da dieses im Inneren verbleibt. Aufgrund der Viskosität des Harzes sollten diese Öffnungen einen Durchmesser von mindestens 3,5mm aufweisen.
	
\end{itemize}


	\newpage
	
	\section{3D-Modell mit PreForm umwandeln und ausdrucken}
 
 \subsection{Preform Vorbereitung}
 
	\begin{itemize}
		\item Programm PreForm öffnen und Drucker "PrettyTiger"\, auswählen. und mit "Auswählen bestätigen.
		\item Mit Klick auf den "Datei"\ Dialog und anschließend "Öffnen"\ STL-Datei öffnen.
		\item Bei Bedarf das Modell mit den Schaltflächen skalieren, drehen und verschieben.
		\item Notwendige Stützstrukturen automatisch erstellen lassen, bei Bedarf händisch anpassen.
		\item Anschließend wird das Teil durch drücken des orangenen Buttons über das Netzwerk an den Drucker gesendet. Hierbei muss ein Name des Drucks eingegeben werden.\\
		\textbf{Namensgebung:} "Max Mustermann\_E-Mail Adresse\_Dateiname"\ 
		
		\item \textbf{Achtung:}Harztank nicht mit Deckel im Drucker lagern, bei Stromanschluss führt der Drucker einen Selbsttest durch und bewegt den Wischer. Alternative: Sicherstellen dass der Drucker nicht angeschaltet werden kann (z.Bsp. Stecker ziehen)
		\end{itemize}
	
	
	\section{Nachbereitung}
	\begin{itemize}
	\item \textbf{Hinweis:}jeder Druck (auch Fehldrucke oder Abbrüche) ist zu wiegen und abzurechnen
	\end{itemize}
	
	\section{Bezahlen und abschließen}
	
	Um das Objekt von der Platte zu lösen vorsichtig arbeiten. Meistens lässt es sich von Hand lösen. Wenn nicht,
	warten bis sich die Platte etwas abgekühlt hat. \textbf{Nicht versuchen, das Objekt mit scharfen oder spitzen Gegenständen herunter zu hebeln!}
	Sollte das Tape oder die Folie auf der Plattform beim Herunterlösen kaputt gehen, bitte einen \textbf{Betreuer es zu erneuern}.
	
	Drucker reinigen, siehe \ref{putzen}.
	
	Objekt mit Feinwaage (steht meist bei den 3D-Druckern) abwiegen, Preis pro Gramm ist im Kassensystem eingetragen.
	
	Es muss alles mitgewogen werden, auch die Stützstruktur und der Müll, den der Extruder anfangs ausspuckt. \textbf{Fehldrucke müssen ebenfalls bezahlt werden.}
	
	 Um den Form 2 in den Ruhestand zu versetzen, drücken und halten Sie die Drucktaste für 6-10 Sekunden, bis die Anzeige erlischt. Betätigen Sie die Drucktaste, um den Drucker aus dem Ruhezustand aufzuwecken.
	\pagebreak
	
	% % % % % % % % % % % %

	\section{Pflege \& Wartung \textcolor{red}{Nur von Betreuer durchzuführen}}
	
	\subsection{Ausrichten}
	Der \textit{Form2} muss, um fehlerfrei drucken zu können, nivelliert sein. Hierzu gibt es ein Eintrag im Einstellungsmenü des Druckers.\\
	Ist der Drucker nicht nivelliert, so ist dieser Menüeintrag auszuwählen und den Anweisungen auf dem Bildschirm zu folgen. Zum justieren wird das runde Nivellierwerkzeug beötigt. Dieses schiebt man unter die jeweilige Ecke und dreht es im Uhrzeigersinn, um den Drucker tiefer, gegen den Uhrzeigersinn, um ihn höher zu stellen.\\
	Weitere Informationen dazu auf der Supportseite des Herstellers.	\\
	 \url{https://support.formlabs.com/hc/de/articles/115000013164-Den-Form-2-nivellierens}
	 
	 \subsection{Pflege der Harzkartusche}
	 Schütteln Sie die Harzkartusche ungefähr alle zwei Wochen, damit das Kunstharz stets gut durchmischt ist. Dadurch erhöht sich die Druckqualität.
	
	\subsection{Harztank}
		Harztank sollte regelmäßig von gereinigt werden. Anleitung dazu auf der Supportseite des Herstellers\\
		\url{https://support.formlabs.com/hc/de/articles/115000016184-Pflege-des-Harztanks}
		
		\subsection{Glasfenster}
		Reinigung des Glasfensters unter dem Harztanks bei Verschmutzung durch Schlieren oder Fingerabdrücken.\\ \url{https://support.formlabs.com/hc/de/articles/115000016204-Reinigung-des-optischen-Glasfensters}
	
	\subsection{Firmwareupdates}
	
In regelmäßigen Abständen kommt es vor, dass die Firmware des Druckers zu updaten ist. Diese Aufgabe sollte ausschließlich von einem Betreuer durchgeführt werden. Eine Anleitung findet sich hierbei auf der Website des Herstellers.\\	
		\url{https://support.formlabs.com/hc/de/articles/115000013690-Firmwareupdate-durchführen}
	
\end{document}
